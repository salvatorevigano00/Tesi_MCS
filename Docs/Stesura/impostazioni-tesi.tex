%*********************************************************************************
% impostazioni-tesi.tex
% di Pierluca Ferraro
% file che contiene le impostazioni della tesi
%*********************************************************************************


%*********************************************************************************
% Comandi personali
%*********************************************************************************
\newcommand{\keyword}[1]{\texttt{#1}}               % Stile per le keyword

\newcommand{\omissis}{[\dots\negthinspace]}         % Omissis
\newcommand{\etal}{\emph{et~al.}}                   % et al.
\newcommand{\HRule}{\rule{\linewidth}{0.2mm}}       % Riga orizzontale
\newcommand{\courierbold}[1]{\usefont{T1}%          % Grassetto monospaced
                {pcr}{b}{n}#1\normalfont}       
%\newcommand\fBold[1]{\textbf #1\relax}             % Primo carattere in grassetto

\newcommand{\figcite}[1]%
  {. Immagine tratta da~\cite{#1}}

\newcommand{\multifigcite}[1]%
  {. Immagini tratte da~\cite{#1}}

\newcommand{\mycaption}[2]{\caption[#1]{#1#2}}


\newcommand{\myTitle}{Titolo tesi}
\newcommand{\myAuthor}{Autore}



%*********************************************************************************
% Comandi per frontespizio e sommario
%*********************************************************************************
\newcommand{\autore}{Dott. Salvatore Vigano'}                 % Autore tesi
\newcommand{\relatore}{Ch.mo Prof. Pierluca Ferraro}           % Relatore
\newcommand{\correlatore}{}               % Correlatore
\newcommand{\controrelatore}{}   % Controrelatore

\newcommand{\university}{%                                   % Università
  \uppercase{Università degli Studi di Palermo}}
\newcommand{\faculty}{%                                      % Facoltà
  \uppercase{Facoltà di Ingegneria}}
\newcommand{\tipolaurea}{%                                   % Tipo di laurea
  \uppercase{\emph{Laurea Magistrale in %
                   Ingegneria Informatica}}}

\newcommand{\titolotesi}{%                                   % Titolo della tesi
  \uppercase{Progettazione e implementazione di un sistema mobile crowdsensing per utenti basati su razionalità limitata}}



%*********************************************************************************
% Impostazioni di geometry e setspace (layout)
%*********************************************************************************
\newcommand{\mybindingoffset}{0cm}                  % Offset rilegatura
%\newcommand{\mybindingoffset}{1cm}                 % Offset rilegatura

\geometry{a4paper,vcentering,                       % Impostazioni margini
          %showframe,
          bindingoffset=\mybindingoffset,
          textwidth=16cm}

\onehalfspacing                                     % Interlinea

%\usepackage{layout}                                % Package per riepilogo layout
%\overfullrule=5pt                                  % Segnala gli overfull hboxes

%*********************************************************************************
% Impostazioni di graphicx
%*********************************************************************************
\graphicspath{{images/}}                            % Path immagini



%*********************************************************************************
% Impostazioni di caption (didascalie)
%*********************************************************************************
\captionsetup{%                                     % Opzioni didascalie
    tableposition=top,%
    figureposition=bottom,%
    font=small,%
    format=hang,%
    labelfont=bf}



%*********************************************************************************
% Impostazioni di sectsty (sezioni personalizzate)
%*********************************************************************************
\allsectionsfont{%                                  % Font di tutte le sezioni
    \usefont{T1}{bch}{b}{n}%                        % Font Avant Garde Bold
%   \hspace{15pt}%                                  % Uncomment for indentation
}                                                   
                                                    
\sectionfont{%                                      % Font del comando \section
    \usefont{T1}{bch}{b}{n}%                        % Font Avant Garde Bold
    %\sectionrule{0pt}{0pt}{-8pt}{0.8pt}%           % Riga orizzontale
}



%*********************************************************************************
% Impostazioni di fancyhdr (header e footer personalizzati)
%*********************************************************************************
\fancyhead[R]{}
\fancyhead[L]{%
  \scshape\footnotesize\nouppercase{\leftmark}}
%\fancyhead[L]{\leftmark}
%\fancyfoot{}                                       % No page footer
%\renewcommand{\headrulewidth}{0pt}                 % Remove header underlines

%\renewcommand{\sectionmark}[1]{%
%  \markboth{}{\footnotesize\usefont{T1}{pag}{m}{n}#1}}

\renewcommand{\chaptermark}[1]{\markboth{{#1}}{}}
\renewcommand{\headrulewidth}{0.3pt}%



\fancypagestyle{firststyle}
{%
\fancyhf{}%
\fancyfoot[C]{\large{Anno Accademico 2025/2026}}%
\renewcommand{\headrulewidth}{0pt}%
\renewcommand{\footrulewidth}{0.3pt}%
}



%*********************************************************************************
% Impostazioni di biblatex
%*********************************************************************************
\defbibheading{bibliografia}{%                      % Heading bibliografia
    \chapter*{Bibliografia}}

\addbibresource{bibliografia.bib}                   % Database della bibliografia

\DeclareCiteCommand*{\citeauthor}                   % Citazione di nome e cognome
{\defcounter{maxnames}{3}%                          % degli autori di un articolo
\defcounter{minnames}{1}%
\defcounter{uniquename}{2}%
\boolfalse{citetracker}%
\boolfalse{pagetracker}%
\usebibmacro{prenote}}
{\ifciteindex{\indexnames{labelname}}{}%
\printnames{labelname}}
{\multicitedelim}
{\usebibmacro{postnote}}



%*********************************************************************************
% Impostazioni di microtype (finezze tipografiche)
%*********************************************************************************
\microtypesetup{final,tracking,kerning,%
                spacing,factor=1100,%
                stretch=10,shrink=10}

\microtypecontext{spacing=nonfrench}


%*********************************************************************************
% Impostazioni di hyperref
%*********************************************************************************
\hypersetup{%
    pdftitle={\myTitle},%
    pdfauthor={\myAuthor},%
    %draft,                                         % Elimina tutti i link
    hidelinks,
    pdfsubject={},%
    pdfkeywords={},%
    pdfcreator={pdfLaTeX},%
    pdfproducer={LaTeX with hyperref}%
}



%*********************************************************************************
% Altro
%*********************************************************************************
\hyphenation{RDF SPARQL RDFS OWL HTML URI}          % Regole per la sillabazione
