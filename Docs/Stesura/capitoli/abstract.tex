Il presente lavoro di tesi si concentra sulla progettazione e validazione di un sistema di \textbf{Mobile Crowdsensing (MCS)} basato sull'algoritmo \textbf{IMCU (Incentive Mechanism for Crowdsensing Users)}, testandone la tenuta su dati reali di mobilità urbana relativi a una flotta di taxi operanti a Roma.
Una prima analisi, condotta ipotizzando utenti perfettamente razionali in un contesto di \textbf{asta inversa veritiera}, conferma la solidità del modello teorico: il sistema risulta stabile e le strategie degli utenti convergono all'equilibrio di Nash. Tuttavia, l'introduzione di scenari più realistici caratterizzati da \textbf{razionalità limitata} --- inclusi errori di valutazione e strategie subottimali --- fa emergere criticità significative. In queste condizioni, l'efficienza del meccanismo IMCU subisce una flessione evidente, rivelando una vulnerabilità strutturale del modello classico.
Per superare tali limiti, il lavoro propone e sperimenta \textbf{GAP}, un meccanismo adattivo che sfrutta l'apprendimento automatico per monitorare il comportamento degli utenti. Modulando dinamicamente incentivi e regole di selezione, GAP permette di ''recuperare'' l'efficienza perduta, ristabilendo la stabilità del sistema anche in contesti comportamentali complessi.
In definitiva, lo studio dimostra che la sostenibilità delle piattaforme di crowdsensing non dipende solo dalla correttezza teorica, ma dalla capacità degli algoritmi di adattarsi elasticamente alla natura imperfetta del comportamento umano.